\documentclass[12pt,letterpaper]{article}
\usepackage[margin=1in]{geometry}
\usepackage{graphicx}
\usepackage{amsmath,amssymb,amsthm}
\usepackage{enumerate,mathtools}
\usepackage{algorithm2e}
\usepackage{tikz}
\usepackage{xcolor}
\usepackage{hyperref}

% Theorems
\usepackage{amsthm}
\renewcommand\qedsymbol{$\blacksquare$}

\newtheorem{definition}{Definition}[section]
\newtheorem{example}{Example}[section]
\newtheorem{theorem}{Theorem}[section]
\newtheorem{corollary}[theorem]{Corollary}
\newtheorem{lemma}[theorem]{Lemma}


% Linear Algebra %

\setlength\unitlength{1mm}

\newcommand{\insertfig}[3]{
\begin{figure}[htbp]\begin{center}\begin{picture}(120,90)
\put(0,-5){\includegraphics[width=12cm,height=9cm,clip=]{#1.eps}}\end{picture}\end{center}
\caption{#2}\label{#3}\end{figure}}

\newcommand{\insertxfig}[4]{
\begin{figure}[htbp]
\begin{center}
\leavevmode \centerline{\resizebox{#4\textwidth}{!}{\input
#1.pstex_t}}
%\vspace*{-0.2in}
\caption{#2} \label{#3}
\end{center}
\end{figure}}

\long\def\comment#1{}

\newcommand\norm[1]{\left\lVert#1\right\rVert}
\DeclareMathOperator*{\argmin}{arg\,min}
\DeclareMathOperator*{\argmax}{arg\,max}
% bb font symbols

\newfont{\bbb}{msbm10 scaled 700}
\newcommand{\CCC}{\mbox{\bbb C}}

\newfont{\bbf}{msbm10 scaled 1100}
\newcommand{\CC}{\mbox{\bbf C}}
\newcommand{\PP}{\mbox{\bbf P}}
\newcommand{\RR}{\mbox{\bbf R}}
\newcommand{\QQ}{\mbox{\bbf Q}}
\newcommand{\ZZ}{\mbox{\bbf Z}}
\renewcommand{\SS}{\mbox{\bbf S}}
\newcommand{\FF}{\mbox{\bbf F}}
\newcommand{\GG}{\mbox{\bbf G}}
\newcommand{\EE}{\mbox{\bbf E}}
\newcommand{\NN}{\mbox{\bbf N}}
\newcommand{\KK}{\mbox{\bbf K}}

% Vectors

\renewcommand{\aa}{{\bf a}}
\newcommand{\bb}{{\bf b}}
\newcommand{\cc}{{\bf c}}
\newcommand{\dd}{{\bf d}}
\newcommand{\ee}{{\bf e}}
\newcommand{\ff}{{\bf f}}
\renewcommand{\gg}{{\bf g}}
\newcommand{\hh}{{\bf h}}
\newcommand{\ii}{{\bf i}}
\newcommand{\jj}{{\bf j}}
\newcommand{\kk}{{\bf k}}
\renewcommand{\ll}{{\bf l}}
\newcommand{\mm}{{\bf m}}
\newcommand{\nn}{{\bf n}}
\newcommand{\oo}{{\bf o}}
\newcommand{\pp}{{\bf p}}
\newcommand{\qq}{{\bf q}}
\newcommand{\rr}{{\bf r}}
\renewcommand{\ss}{{\bf s}}
\renewcommand{\tt}{{\bf t}}
\newcommand{\uu}{{\bf u}}
\newcommand{\ww}{{\bf w}}
\newcommand{\vv}{{\bf v}}
\newcommand{\xx}{{\bf x}}
\newcommand{\yy}{{\bf y}}
\newcommand{\zz}{{\bf z}}
\newcommand{\0}{{\bf 0}}
\newcommand{\1}{{\bf 1}}

% Matrices

\newcommand{\Ab}{{\bf A}}
\newcommand{\Bb}{{\bf B}}
\newcommand{\Cb}{{\bf C}}
\newcommand{\Db}{{\bf D}}
\newcommand{\Eb}{{\bf E}}
\newcommand{\Fb}{{\bf F}}
\newcommand{\Gb}{{\bf G}}
\newcommand{\Hb}{{\bf H}}
\newcommand{\Ib}{{\bf I}}
\newcommand{\Jb}{{\bf J}}
\newcommand{\Kb}{{\bf K}}
\newcommand{\Lb}{{\bf L}}
\newcommand{\Mb}{{\bf M}}
\newcommand{\Nb}{{\bf N}}
\newcommand{\Ob}{{\bf O}}
\newcommand{\Pb}{{\bf P}}
\newcommand{\Qb}{{\bf Q}}
\newcommand{\Rb}{{\bf R}}
\newcommand{\Sb}{{\bf S}}
\newcommand{\Tb}{{\bf T}}
\newcommand{\Ub}{{\bf U}}
\newcommand{\Wb}{{\bf W}}
\newcommand{\Vb}{{\bf V}}
\newcommand{\Xb}{{\bf X}}
\newcommand{\Yb}{{\bf Y}}
\newcommand{\Zb}{{\bf Z}}

% Calligraphic

\newcommand{\Ac}{{\cal A}}
\newcommand{\Bc}{{\cal B}}
\newcommand{\Cc}{{\cal C}}
\newcommand{\Dc}{{\cal D}}
\newcommand{\Ec}{{\cal E}}
\newcommand{\Fc}{{\cal F}}
\newcommand{\Gc}{{\cal G}}
\newcommand{\Hc}{{\cal H}}
\newcommand{\Ic}{{\cal I}}
\newcommand{\Jc}{{\cal J}}
\newcommand{\Kc}{{\cal K}}
\newcommand{\Lc}{{\cal L}}
\newcommand{\Mc}{{\cal M}}
\newcommand{\Nc}{{\cal N}}
\newcommand{\Oc}{{\cal O}}
\newcommand{\Pc}{{\cal P}}
\newcommand{\Qc}{{\cal Q}}
\newcommand{\Rc}{{\cal R}}
\newcommand{\Sc}{{\cal S}}
\newcommand{\Tc}{{\cal T}}
\newcommand{\Uc}{{\cal U}}
\newcommand{\Wc}{{\cal W}}
\newcommand{\Vc}{{\cal V}}
\newcommand{\Xc}{{\cal X}}
\newcommand{\Yc}{{\cal Y}}
\newcommand{\Zc}{{\cal Z}}

% Bold greek letters

\newcommand{\alphab}{\hbox{\boldmath$\alpha$}}
\newcommand{\betab}{\hbox{\boldmath$\beta$}}
\newcommand{\gammab}{\hbox{\boldmath$\gamma$}}
\newcommand{\deltab}{\hbox{\boldmath$\delta$}}
\newcommand{\etab}{\hbox{\boldmath$\eta$}}
\newcommand{\lambdab}{\hbox{\boldmath$\lambda$}}
\newcommand{\epsilonb}{\hbox{\boldmath$\epsilon$}}
\newcommand{\nub}{\hbox{\boldmath$\nu$}}
\newcommand{\mub}{\hbox{\boldmath$\mu$}}
\newcommand{\zetab}{\hbox{\boldmath$\zeta$}}
\newcommand{\phib}{\hbox{\boldmath$\phi$}}
\newcommand{\psib}{\hbox{\boldmath$\psi$}}
\newcommand{\thetab}{\hbox{\boldmath$\theta$}}
\newcommand{\taub}{\hbox{\boldmath$\tau$}}
\newcommand{\omegab}{\hbox{\boldmath$\omega$}}
\newcommand{\xib}{\hbox{\boldmath$\xi$}}
\newcommand{\sigmab}{\hbox{\boldmath$\sigma$}}
\newcommand{\pib}{\hbox{\boldmath$\pi$}}
\newcommand{\rhob}{\hbox{\boldmath$\rho$}}

\newcommand{\Gammab}{\hbox{\boldmath$\Gamma$}}
\newcommand{\Lambdab}{\hbox{\boldmath$\Lambda$}}
\newcommand{\Deltab}{\hbox{\boldmath$\Delta$}}
\newcommand{\Sigmab}{\hbox{\boldmath$\Sigma$}}
\newcommand{\Phib}{\hbox{\boldmath$\Phi$}}
\newcommand{\Pib}{\hbox{\boldmath$\Pi$}}
\newcommand{\Psib}{\hbox{\boldmath$\Psi$}}
\newcommand{\Thetab}{\hbox{\boldmath$\Theta$}}
\newcommand{\Omegab}{\hbox{\boldmath$\Omega$}}
\newcommand{\Xib}{\hbox{\boldmath$\Xi$}}


% mixed symbols

\newcommand{\sinc}{{\hbox{sinc}}}
\newcommand{\diag}{{\hbox{diag}}}
\renewcommand{\det}{{\hbox{det}}}
\newcommand{\trace}{{\hbox{tr}}}
\newcommand{\tr}{\trace}
\newcommand{\sign}{{\hbox{sign}}}
\renewcommand{\arg}{{\hbox{arg}}}
\newcommand{\var}{{\hbox{var}}}
\newcommand{\cov}{{\hbox{cov}}}
\renewcommand{\Re}{{\rm Re}}
\renewcommand{\Im}{{\rm Im}}
\newcommand{\eqdef}{\stackrel{\Delta}{=}}
\newcommand{\defines}{{\,\,\stackrel{\scriptscriptstyle \bigtriangleup}{=}\,\,}}
\newcommand{\<}{\left\langle}
\renewcommand{\>}{\right\rangle}
\newcommand{\Psf}{{\sf P}}
\newcommand{\T}{\top}
\newcommand{\m}[1]{\begin{bmatrix} #1 \end{bmatrix}}


\begin{document}

\begin{center}
    \large Constrained Optimization

    \normalsize Conner DiPaolo\footnote{Much of this material originates from
    Boyd \& Vandenberghe's \textit{Convex Optimization}. If you're looking for
    a more detailed analysis of these methods, consult that text. It's free online
    at \url{http://stanford.edu/~boyd/cvxbook/}}
\end{center}
\vspace{3\baselineskip}

\section{Introduction}

We have already covered methods used to solve convex optimization problems of
the form
\begin{align*}
    \text{minimize: } & f(\xx)\\
\end{align*}
where $f$ is convex. This problem is called \textit{unconstrained}, since the place
on constraints on $\xx$, besides that it must live in $\RR^n$. Namely, one of the
quickest and most consistent algorithms for solving such problems (ignoring some aspects
of scaling) is \textit{Newton's Method}, where we approximate our objective as a quadratic
using the second order Taylor approximation $f(\xx) \approx f(\xx_i) + \nabla f(\xx_i)(\xx - \xx_i)
+ \dfrac{1}{2}(\xx - \xx_i)^\T\nabla^2 f(\xx_i) (\xx - \xx_i)$ and solve that in closed form
to find the next iterate $\xx_{i+1}$. Newton's method is \textit{fast}, obeying a property
called \textit{quadratic convergence}. Once $\xx_i$ gets close enough (for some unknown measure
of `enough') to the optimal solution, every Newton iteration will double the number of significant
digits in your current approximation of the optimal $\xx^\star$. When you are not close to
the optimal solution, Newton's method linearly converges to the optimal $\xx^\star$ (under some conditions
including convexity). We have never addressed, up until this point, how to solve more general convex
optimization problems which can be \textit{constrained}
\begin{align*}
    \text{minimize: } & f_i(\xx)        \tag{objective function}\\
    \text{subj. to: } & f_i(\xx) \leq 0 \tag{inequality constraints}\\
                      & A\xx = \bb.     \tag{equality constraints}
\end{align*}
For this problem to be convex we require that the feasible set (the region of $\RR^n$ where the
constraints hold) be a convex set, and that the objective function be convex. This implies that
all inequality constraints $f_i$ must be convex, and that equality constraints must be affine.
We will first attack how to solve equality constrained convex optimization problems before
moving onto inequality constrained problems. Combining the two methods is trivial once you have
both, and we will briefly address this.

\section{Equality Constrained Minimization}
\subsection{Equality Constrained Newton's Method}

For unconstrained Newton's method, at every iteration we approximated our objective as a quadratic
about our current estimate for the optimal $\xx^\star$, and found a new approximate optimum under
that quadratic approximation. It would make sense, then, for equality constrained optimization problems
of the form
\begin{align*}
    \text{minimize: } & f(\xx)\\
    \text{subj. to: } & A\xx = b
\end{align*}
to just approximate the objective as a quadratic at each iteration and solve that problem in closed form.
Is a quadratic program with equality constraints solvable in closed form? Yes!

\begin{example}[Equality Constrained Convex Quadratic Minimization]
    Consider the task of solving the problem
    \begin{align*}
        \text{minimize: } & \frac{1}{2}\xx^\T P\xx + \qq^\T\xx + \rr\\
        \text{subj. to: } & A\xx = \bb.
    \end{align*}
    where $P \succeq 0$. From this condition on $P$ we know the problem is convex.
    From the KKT conditions we know at optimality
    \[
        A\xx^\star = \bb
    \]
    to ensure primal feasibility.
    Further, we can see that this has a Lagrangian
    \[
        \Lc(\xx, \nub) = \frac{1}{2}\xx^\T P\xx + \qq^\T\xx + \rr + (A\xx - \bb)^\T\nub.
    \]
    From the KKT conditions, again, we know that $\nabla L = 0$
    at optimality, so
    \[
        \nabla_{\xx} \Lc(\xx^\star, \nub^\star) = P\xx^\star + \qq + A^\T\nub^\star = 0.
    \]
    Thus we have two equations in two unknowns, and can solve this
    system analytically in the block form
    \[
        \m{P & A^\T \\ A & 0} \m{\xx^\star \\ \nub^\star} = \m{-\qq \\ \bb}
    \]
    This is known as the KKT system, and can be solved as any linear system would.
\end{example}

So now let's attack the general equality constrained problem in a Newton-esque algorithm
that we'll call Equality Constrained Newton's Method. First, create our approximate problem
\begin{align*}
    \text{minimize: } & f(\xx_i) + \nabla f(\xx_i) (\xx - \xx_i) + \frac{1}{2}(\xx - \xx_i)^\T\nabla^2 f(\xx_i) (\xx - \xx_i)\\
    \text{subj. to: } & A\xx = \bb.
\end{align*}
This is close to the form of the example above, but we need $A\xx = \bb$ to be of
the form $A(\xx - \xx_i) = \cc$ for some $\cc$. Assuming $A\xx = \bb$ to begin with,
this gives the problem
\begin{align*}
    \text{minimize: } & f(\xx_i) + \nabla f(\xx_i) (\xx - \xx_i) + \frac{1}{2}(\xx - \xx_i)^\T\nabla^2 f(\xx_i) (\xx - \xx_i)\\
    \text{subj. to: } & A(\xx-\xx_i) = \0.
\end{align*}
Denote $\Delta \xx_i = \xx_{i+1} - \xx_i$. Then at each step we solve the KKT system
\[
    \m{\nabla^2 f(\xx_i) & A^\T\\A & 0} \m{\Delta \xx_i\\\nub^\star} = \m{-\nabla f(\xx_i)\\\0}
\]
for $\Delta\xx_i$ and give the next iterate $\xx_{i+1} = \Delta\xx_i + \xx_i$, repeating this process.
Note that this algorithm reduces to a regular Newton step if we have no equality constraints. In a cleaner
form we have the following algorithm.

\IncMargin{1em}
\begin{algorithm}
    \SetKwInOut{Input}{input}\SetKwInOut{Output}{output}
    \textbf{Equality Constrained Newton's Method}\\
    \Input{$f$, $\nabla f$, $\nabla^2f$, starting point $\xx_0$ such that $A\xx_0 = \bb$, tolerance $\epsilon$}
    \Output{optimal point $\xx^\star$}
    $i \leftarrow 0$\\
    \While{$|| \Delta\xx_i ||_2 \geq \epsilon$}{
        solve the KKT system for $\Delta\xx_i$\\
        $\xx_{i+1} \leftarrow \xx_i + \Delta\xx_i$\\
        $i \leftarrow i + 1$
    }
    \textbf{return } $\xx^\star = \xx_{i+1}$
\end{algorithm}

\subsection{Infeasible Start Equality Constrained Newton's Method}

One drawback of using the above algorithm is that we must start with some $\xx_0$ such that
$A\xx_0 = \bb$, which could be nontrivial to compute. It would be better if that step of generating
feasibility is encapsulated in our algorithm. Now suppose that $A\xx_0 \neq b$. Then to ensure primal
feasibility in each step our original equality constraint $A(\xx + \Delta\xx) = \bb$ gives
$A\Delta\xx = \bb - A\xx$. Note that if we have feasible $\xx$ the equality constraint reduces to
$A\Delta\xx = \0$ as seen above in the feasible start method. Substituting this into the optimality
conditions gives the revised KKT system
\[
    \m{\nabla^2 f(\xx_i) & A^\T\\A & 0} \m{\Delta \xx_i\\\nub^\star} = -\m{\nabla f(\xx_i)\\A\xx - \bb}.
\]
Using this new system to solve for $\Delta\xx_i$ gives a feasible point $\xx_i + \Delta\xx_i$ after any iteration.
This, then, gives a slightly more general algorithm, Infeasible Start Newton's Method.

\IncMargin{1em}
\begin{algorithm}
    \SetKwInOut{Input}{input}\SetKwInOut{Output}{output}
    \textbf{Infeasible Start Newton's Method}\\
    \Input{$f$, $\nabla f$, $\nabla^2f$, starting point $\xx_0$, tolerance $\epsilon$}
    \Output{optimal point $\xx^\star$}
    $i \leftarrow 0$\\
    \While{$|| \Delta\xx_i ||_2 \geq \epsilon$}{
        solve the infeasible start KKT system for $\Delta\xx_i$\\
        $\xx_{i+1} \leftarrow \xx_i + \Delta\xx_i$\\
        $i \leftarrow i + 1$
    }
    \textbf{return } $\xx^\star = \xx_{i+1}$
\end{algorithm}

\section{General Constrained Minimization}

Consider the convex problem
\begin{align*}
    \text{minimize: } & f_0(\xx)\\
    \text{subj. to: } & f_i(\xx) \leq 0 \qquad i = 1,2,\dots,m\\
                      & A\xx = \bb.
\end{align*}
One way to turn this into a unconstrained optimization problem, taking motivation from the Lagrangian,
would be to minimize the objective, but when the constraints aren't satisfied to bump the objective
$f_0$ to infinity. We can do this with the indicator function
\[
I_-(u) = \begin{cases*}0 & $u \leq 0$\\ \infty & otherwise.\end{cases*}
\]
This gives the equivalent optimization problem
\begin{align*}
    \text{minimize: } & f_0(\xx) + \sum_{i=1}^m I_-(f_i(\xx))\\
    \text{subj. to: } & A\xx = \bb
\end{align*}
This is great, but we can't differentiate the objective and therefore can't use Infeasible Start Newton Method
shown above to solve this problem. Instead, we approximate
\[
    I_-(u) \approx -\frac{1}{t}\log(-u).
\]
As $t$ approaches $\infty$, this approximate converges to $I_-$ pointwise. This gives the
\textit{differentiable} optimization problem (assuming $f_0$ is differentiable, of course)
\begin{align*}
    \text{minimize: } & f_0(\xx) - \frac{1}{t}\sum_{i=1}^m \log(-f_i(\xx))\\
    \text{subj. to: } & A\xx = \bb
\end{align*}
Since the negative $\log$ of a convex function is convex, this problem also retains convexity, and we can
solve it using Infeasible Start Newton Method! Call the solution to the above problem $\xx_t^\star$. We can
show (and Boyd and Vandenberghe do!) that $\xx_t^\star$ is $m/t$ suboptimal. Then if we want $\epsilon$ accuracy
we could just set $t = m/\epsilon$ and solve that problem:
\begin{align*}
    \text{minimize: } & f_0(\xx) - \frac{\epsilon}{m}\sum_{i=1}^m \log(-f_i(\xx))\\
    \text{subj. to: } & A\xx = \bb
\end{align*}
This would work in theory, but practical issues with $\epsilon$ being too small, for example, causes this to not work
well. We can change this slightly to work very well in practice. Instead of just starting with some large $t$,
we solve a sequence of smaller problems $\xx_{t_0}^\star,\xx_{t_1}^\star,\dots$ for increasing $t_i$ until
$t_i \geq m/\epsilon$, ensuring an accuracy $\epsilon$. This algorithm is outlined as follows.

\IncMargin{1em}
\begin{algorithm}
    \SetKwInOut{Input}{input}\SetKwInOut{Output}{output}
    \textbf{Barrier Method}\\
    \Input{strictly feasible $\xx$, $t := t^{(0)}>0$, $\mu>1$, tolerance $\epsilon>0$}
    \Output{optimal point $\xx^\star$}
    $i \leftarrow 0$\\
    \While{$m/t < \epsilon$}{
        compute $\xx_{t^(i)}^\star$ using $\xx_{t^{i-1}}^\star$ as a starting point\\
        set $t \leftarrow \mu t$
    }
    \textbf{return } $\xx^\star = \xx_{t^{(i+1)}}^\star$
\end{algorithm}

Boyd and Vandenberghe go into a detailed analysis of why this is true, but in practice
setting $\mu$ between $3$ and $100$ or so seems to work well. Basically, if $\mu$ is small,
each problem iterate is likely to be close to the previous problem, and we will hug the
quadratic convergence of Newton's Method closely, resulting in few inner iterations but many
outer iterations. If $\mu$ is large, the problem iterates are likely to be quite different
from the previous iterates, which results in many inner iterations to solve the problems but
few outer iterations. Empirically, when $\mu$ is between $3$ and $100$, these effects tend to
cancel one another out.\\

The Barrier Method is what is known as an \textit{Interior Point Method}, because it traverses
the interior of the feasible set to find the optimal solution (when iterating over $t$) instead
of hugging the boundary of the region, like the Simplex algorithm for linear programs does. With
this algorithm, you should be able to solve all optimization problems we would deal ever deal with
in the course.

\end{document}

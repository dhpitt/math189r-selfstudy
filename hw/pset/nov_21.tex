\documentclass[12pt,letterpaper,fleqn]{hmcpset}
\usepackage[margin=1in]{geometry}
\usepackage{graphicx}
\usepackage{amsmath,amssymb}
\usepackage{enumerate}
\usepackage{hyperref}
\usepackage{parskip}

% Theorems
\usepackage{amsthm}
\renewcommand\qedsymbol{$\blacksquare$}

\newtheorem{definition}{Definition}[section]
\newtheorem{example}{Example}[section]
\newtheorem{theorem}{Theorem}[section]
\newtheorem{corollary}[theorem]{Corollary}
\newtheorem{lemma}[theorem]{Lemma}


% Linear Algebra %

\setlength\unitlength{1mm}

\newcommand{\insertfig}[3]{
\begin{figure}[htbp]\begin{center}\begin{picture}(120,90)
\put(0,-5){\includegraphics[width=12cm,height=9cm,clip=]{#1.eps}}\end{picture}\end{center}
\caption{#2}\label{#3}\end{figure}}

\newcommand{\insertxfig}[4]{
\begin{figure}[htbp]
\begin{center}
\leavevmode \centerline{\resizebox{#4\textwidth}{!}{\input
#1.pstex_t}}
%\vspace*{-0.2in}
\caption{#2} \label{#3}
\end{center}
\end{figure}}

\long\def\comment#1{}

\newcommand\norm[1]{\left\lVert#1\right\rVert}
\DeclareMathOperator*{\argmin}{arg\,min}
\DeclareMathOperator*{\argmax}{arg\,max}
% bb font symbols

\newfont{\bbb}{msbm10 scaled 700}
\newcommand{\CCC}{\mbox{\bbb C}}

\newfont{\bbf}{msbm10 scaled 1100}
\newcommand{\CC}{\mbox{\bbf C}}
\newcommand{\PP}{\mbox{\bbf P}}
\newcommand{\RR}{\mbox{\bbf R}}
\newcommand{\QQ}{\mbox{\bbf Q}}
\newcommand{\ZZ}{\mbox{\bbf Z}}
\renewcommand{\SS}{\mbox{\bbf S}}
\newcommand{\FF}{\mbox{\bbf F}}
\newcommand{\GG}{\mbox{\bbf G}}
\newcommand{\EE}{\mbox{\bbf E}}
\newcommand{\NN}{\mbox{\bbf N}}
\newcommand{\KK}{\mbox{\bbf K}}

% Vectors

\renewcommand{\aa}{{\bf a}}
\newcommand{\bb}{{\bf b}}
\newcommand{\cc}{{\bf c}}
\newcommand{\dd}{{\bf d}}
\newcommand{\ee}{{\bf e}}
\newcommand{\ff}{{\bf f}}
\renewcommand{\gg}{{\bf g}}
\newcommand{\hh}{{\bf h}}
\newcommand{\ii}{{\bf i}}
\newcommand{\jj}{{\bf j}}
\newcommand{\kk}{{\bf k}}
\renewcommand{\ll}{{\bf l}}
\newcommand{\mm}{{\bf m}}
\newcommand{\nn}{{\bf n}}
\newcommand{\oo}{{\bf o}}
\newcommand{\pp}{{\bf p}}
\newcommand{\qq}{{\bf q}}
\newcommand{\rr}{{\bf r}}
\renewcommand{\ss}{{\bf s}}
\renewcommand{\tt}{{\bf t}}
\newcommand{\uu}{{\bf u}}
\newcommand{\ww}{{\bf w}}
\newcommand{\vv}{{\bf v}}
\newcommand{\xx}{{\bf x}}
\newcommand{\yy}{{\bf y}}
\newcommand{\zz}{{\bf z}}
\newcommand{\0}{{\bf 0}}
\newcommand{\1}{{\bf 1}}

% Matrices

\newcommand{\Ab}{{\bf A}}
\newcommand{\Bb}{{\bf B}}
\newcommand{\Cb}{{\bf C}}
\newcommand{\Db}{{\bf D}}
\newcommand{\Eb}{{\bf E}}
\newcommand{\Fb}{{\bf F}}
\newcommand{\Gb}{{\bf G}}
\newcommand{\Hb}{{\bf H}}
\newcommand{\Ib}{{\bf I}}
\newcommand{\Jb}{{\bf J}}
\newcommand{\Kb}{{\bf K}}
\newcommand{\Lb}{{\bf L}}
\newcommand{\Mb}{{\bf M}}
\newcommand{\Nb}{{\bf N}}
\newcommand{\Ob}{{\bf O}}
\newcommand{\Pb}{{\bf P}}
\newcommand{\Qb}{{\bf Q}}
\newcommand{\Rb}{{\bf R}}
\newcommand{\Sb}{{\bf S}}
\newcommand{\Tb}{{\bf T}}
\newcommand{\Ub}{{\bf U}}
\newcommand{\Wb}{{\bf W}}
\newcommand{\Vb}{{\bf V}}
\newcommand{\Xb}{{\bf X}}
\newcommand{\Yb}{{\bf Y}}
\newcommand{\Zb}{{\bf Z}}

% Calligraphic

\newcommand{\Ac}{{\cal A}}
\newcommand{\Bc}{{\cal B}}
\newcommand{\Cc}{{\cal C}}
\newcommand{\Dc}{{\cal D}}
\newcommand{\Ec}{{\cal E}}
\newcommand{\Fc}{{\cal F}}
\newcommand{\Gc}{{\cal G}}
\newcommand{\Hc}{{\cal H}}
\newcommand{\Ic}{{\cal I}}
\newcommand{\Jc}{{\cal J}}
\newcommand{\Kc}{{\cal K}}
\newcommand{\Lc}{{\cal L}}
\newcommand{\Mc}{{\cal M}}
\newcommand{\Nc}{{\cal N}}
\newcommand{\Oc}{{\cal O}}
\newcommand{\Pc}{{\cal P}}
\newcommand{\Qc}{{\cal Q}}
\newcommand{\Rc}{{\cal R}}
\newcommand{\Sc}{{\cal S}}
\newcommand{\Tc}{{\cal T}}
\newcommand{\Uc}{{\cal U}}
\newcommand{\Wc}{{\cal W}}
\newcommand{\Vc}{{\cal V}}
\newcommand{\Xc}{{\cal X}}
\newcommand{\Yc}{{\cal Y}}
\newcommand{\Zc}{{\cal Z}}

% Bold greek letters

\newcommand{\alphab}{\hbox{\boldmath$\alpha$}}
\newcommand{\betab}{\hbox{\boldmath$\beta$}}
\newcommand{\gammab}{\hbox{\boldmath$\gamma$}}
\newcommand{\deltab}{\hbox{\boldmath$\delta$}}
\newcommand{\etab}{\hbox{\boldmath$\eta$}}
\newcommand{\lambdab}{\hbox{\boldmath$\lambda$}}
\newcommand{\epsilonb}{\hbox{\boldmath$\epsilon$}}
\newcommand{\nub}{\hbox{\boldmath$\nu$}}
\newcommand{\mub}{\hbox{\boldmath$\mu$}}
\newcommand{\zetab}{\hbox{\boldmath$\zeta$}}
\newcommand{\phib}{\hbox{\boldmath$\phi$}}
\newcommand{\psib}{\hbox{\boldmath$\psi$}}
\newcommand{\thetab}{\hbox{\boldmath$\theta$}}
\newcommand{\taub}{\hbox{\boldmath$\tau$}}
\newcommand{\omegab}{\hbox{\boldmath$\omega$}}
\newcommand{\xib}{\hbox{\boldmath$\xi$}}
\newcommand{\sigmab}{\hbox{\boldmath$\sigma$}}
\newcommand{\pib}{\hbox{\boldmath$\pi$}}
\newcommand{\rhob}{\hbox{\boldmath$\rho$}}

\newcommand{\Gammab}{\hbox{\boldmath$\Gamma$}}
\newcommand{\Lambdab}{\hbox{\boldmath$\Lambda$}}
\newcommand{\Deltab}{\hbox{\boldmath$\Delta$}}
\newcommand{\Sigmab}{\hbox{\boldmath$\Sigma$}}
\newcommand{\Phib}{\hbox{\boldmath$\Phi$}}
\newcommand{\Pib}{\hbox{\boldmath$\Pi$}}
\newcommand{\Psib}{\hbox{\boldmath$\Psi$}}
\newcommand{\Thetab}{\hbox{\boldmath$\Theta$}}
\newcommand{\Omegab}{\hbox{\boldmath$\Omega$}}
\newcommand{\Xib}{\hbox{\boldmath$\Xi$}}


% mixed symbols

\newcommand{\sinc}{{\hbox{sinc}}}
\newcommand{\diag}{{\hbox{diag}}}
\renewcommand{\det}{{\hbox{det}}}
\newcommand{\trace}{{\hbox{tr}}}
\newcommand{\tr}{\trace}
\newcommand{\sign}{{\hbox{sign}}}
\renewcommand{\arg}{{\hbox{arg}}}
\newcommand{\var}{{\hbox{var}}}
\newcommand{\cov}{{\hbox{cov}}}
\renewcommand{\Re}{{\rm Re}}
\renewcommand{\Im}{{\rm Im}}
\newcommand{\eqdef}{\stackrel{\Delta}{=}}
\newcommand{\defines}{{\,\,\stackrel{\scriptscriptstyle \bigtriangleup}{=}\,\,}}
\newcommand{\<}{\left\langle}
\renewcommand{\>}{\right\rangle}
\newcommand{\Psf}{{\sf P}}
\newcommand{\T}{\top}
\newcommand{\m}[1]{\begin{bmatrix} #1 \end{bmatrix}}


% info for header block in upper right hand corner
\name{------}
\class{Math 189r}
\assignment{Homework 3}
\duedate{November 21, 2016}

\begin{document}

There are 5 problems in this set. You need to do 3 problems the first week and 2 the second
week. Instead of a sixth problem, \textbf{you are encouraged to work on your final project}.
Feel free to work with other students, but make sure you write up the homework
and code on your own (no copying homework \textit{or} code; no pair programming).
Feel free to ask students or instructors for help debugging code or whatever else,
though.
When implementing algorithms you may not use any library (such as \texttt{sklearn})
that already implements the algorithms but you may use any other library for
data cleaning and numeric purposes (\texttt{numpy} or \texttt{pandas}). Use common
sense. Problems are in no specific order.\\[1em]

\textbf{1 (Gaussian Mixture Model)} Consider the generative process for a Gaussian
Mixture Model:
\begin{enumerate}[(1)]
    \item Draw $z_i \sim \mathrm{Cat}(\pib)$ for $i=1,2,\dots,n$.
    \item Draw $\xx_i \sim \Nc(\mub_{z_i}, \Sigmab_{z_i})$ for $i=1,2,\dots,n$.
\end{enumerate}
Note that $z_i$ is unobserved but $\xx_i$ is observed.
Express this model as a directed graphical model, first `unrolled' and then using
Plate notation, before answering the following questions. Support all claims.
\begin{enumerate}[(a)]
    \item Is $\pib$ independent of $\mub_{z_i}$ or $\Sigmab_{z_i}$ given
        your dataset $\Dc = \{\xx_i\}$? Does the posterior distribution over
        $\{\mub,\Sigmab\}$ and $\pib$ factorize? How does this change what inference
        procedure we need to use for this model?
    \item If $z_i$ were observed, would this change? Would the posterior then
        factorize? \textit{Hint:} what other model have we studied that corresponds to
        observing $z_i$?
    \item Find the maximum likelihood estimates for $\pib$, $\mub_k$, and $\Sigmab_k$
        \textit{if} the latent variables $z_i$ were observed.\\
\end{enumerate}

\textbf{2 (Linear Regression)} Consider the Bayesian Linear Regression model with
the following generative process:
\begin{enumerate}[(1)]
    \item Draw $\ww \sim \Nc(\0, \mathbf{V}_0)$
    \item Draw $\yy_i \sim \Nc(\ww^\T\xx_i, \sigma^2)$ for $i=1,2,\dots,n$ where $\sigma^2$
        is known.
\end{enumerate}
Express this model as a directed graphical model using Plate notation. Is $\yy_i$
independent of $\ww$? Is $\yy_i$ independent of $\ww$ \textit{given} $\Dc = \{\xx_i\}$? Support
these claims.\newpage

\textbf{3 (Collaborative Filtering)} Consider the `ratings' matrix $\Rb\in\RR^{n\times n}$
with the low rank approximation $\Rb = \Ub \Vb^\T$ where $\Ub$ and $\Vb$ live in
$\RR^{n\times k}$ where we have $k$ latent factors. Define our optimization problem as
\begin{align*}
    \text{minimize: } & f(\Ub,\Vb) = \|\Rb - \Ub\Vb^\T\|_2^2 + \lambda\|\Ub\|_2^2 + \gamma\|\Vb\|_2^2
\end{align*}
where $\|\cdot\|_2$ in this case is the Frobenius norm $\|\Rb\|_2^2 = \sum_{ij}\Rb_{ij}^2$.
Derive the gradient of $f$ with respect to $\Ub_i$ and $\Vb_j$. Derive a stochastic
approximation to this gradient where you consider a single data point at a time.\\

\textbf{4 (Alternating Least Squares)} Consider the same setup and objective
\begin{align*}
    \text{minimize: } & f(\Ub,\Vb) = \|\Rb - \Ub\Vb^\T\|_2^2 + \lambda\|\Ub\|_2^2 + \gamma\|\Vb\|_2^2
\end{align*}
as above in problem (3).
\begin{enumerate}[(a)]
    \item Suppose we fix $\Ub$. Find the optimal $\Vb$.
    \item Suppose we fix $\Vb$. Find the optimal $\Ub$.
    \item Propose an EM-like (block coordinate ascent, to be exact) like algorithm
        for minimizing $f(\Ub,\Vb)$ using (a) and (b).
    \item Will the algorithm you propose in (c) necessarily converge to the global
        optimal?
\end{enumerate}

\textbf{5 (Non-Negative Matrix Factorization)} Consider the dataset at
\url{http://kdd.ics.uci.edu/databases/reuters21578/reuters21578.html}. Choosing an appropriate
objective function and algorithm from Lee and Seung 2001\footnote{\url{https://papers.nips.cc/paper/1861-algorithms-for-non-negative-matrix-factorization.pdf}}
implement Non-Negative Matrix Factorization for topic modelling (choose an appropriate number
of topics/latent features) and assert that the convergence properties proved in the paper hold. 
Display the 20 most relevant words for each of the topics you discover.

\end{document}
